\documentclass{beamer}
\usepackage{graphicx}
\usepackage[brazil]{babel}
\usepackage[utf8]{inputenc}

\title{Apresentação}
\author{Diogo Moura}
\date{\today}

\usetheme{Hannover}

\begin{document}

\frame{\titlepage}
\frame{
  \frametitle{Sumario}
  \tableofcontents}

\section{Introdução}
\begin{frame}
  \frametitle{Introdução ao modelo de Apresentação}
  \begin{block}{Exemplo de bloco 1}
    Aqui vai o texto do bloco 1
\end{block}

\begin{block}{Exemplo de bloco 2}
  Aqui vai o texto do bloco 2
\end{block}
\end{frame}

\section{Corpo}
\begin{frame}
  \frametitle{Desenvolvimento}
  \begin{columns}[t]
    \begin{column}{5cm}
      aqui vai o texto da primeira coluna
      
    \end{column}

    aqui vai o texto da segunda coluna
    
  \end{columns}
  

\end{frame}

\begin{frame}
  \frametitle{Desenvolvimento parte 2}
  \begin{columns}[t]
    \begin{column}{5cm}
      \begin{block}{exemplo}

        Uso de bloco em colunas
        
      \end{block}
      aqui vai o texto da primeira coluna
      
    \end{column}

    aqui vai o texto da segunda coluna
    
  \end{columns}
\end{frame}

\begin{frame}{Caixas}
  \begin{center}
    \begin{tabular}{|c|}
      \hline 
      Quando quiser adicionar uma quantidade maior de texto,\\ pode-se usar o ambiente tabular\\
      \hline
    \end{tabular}
  \end{center}
  \footnote{Notas de rodapé}
\end{frame}

\begin{frame}{Transição}
  \transdissolve
  \begin{itemize}
    \item item 1 \pause
    \item item 2 \pause
    \item item 3 
    
  \end{itemize}
\end{frame}

\end{document}