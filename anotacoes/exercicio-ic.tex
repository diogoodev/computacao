\documentclass{article}
\usepackage{graphicx}
\usepackage[brazil]{babel}
\usepackage[utf8]{inputenc}
\usepackage[normalem]{ulem}

\title{Exemplo de Aula}
\author{Diogo Moura}
\date{\today}

\begin{document}
\maketitle
\section{Introdução}
Knuth, no seu livro de \TeX{},
faz diferencia o pessoal
que trabalha com \TeX{} em
\TeX{}nicos e \TeX perto

Para usar tem que \textsl{estudar}!
\subsection{Comandos Básicos}
Aqui aprenderemos os comandos básicos do latex.

\centering{Comandos Iniciais}

\flushleft{ a esquerda}

{\scriptsize muito pequeno}

{\footnotesize menor}

{\small pequeno}

{\large grande}

{\LARGE maior ainda}

{\huge maior}

{\Huge maior de todos}

\emph{enfase}

\flushright{ a direita}
  
{\tiny o menor}

\textit{itálico}

\textbf{negrito}

\textrm{romano}

\textsf{sans serif}

\texttt{Maquina de escrever}

\textsc{caixa alta}

\uline{sublinhado}

\uuline{duplo sublinhado}

\uwave{sublinhado curvo}

\sout{riscado}

\xout{muito riscado}

\flushleft{}
\subsection{listas}

Iremos iniciar pelas listas e descrições:
\begin{enumerate}
  \item Lista 1:
  \begin{itemize}
    \item Conteúdo 1
    \item[-] Conteúdo 2
  \end{itemize}
  \item Lista 2:
  \item Lista 3:
\end{enumerate}

\subsection*{tabelas}
Exemplos de uso de tabelas
\begin{table}
  
  \centering
  \begin{tabular}[c]{r|l}
    \hline
    7C0 & HEXADECIMAL \\
    7C3 & HEXADECIMAL \\
    3700 & OCTAL \\ \cline{2 - 2}
    1111010101 & BINÁRIO \\
    \hline \hline
    0.1984 & decimal \\
    \hline
    2 & 1\\ 
    &
    
  \end{tabular}
  \caption{Tabela 1}
  \label{tab:mylabel}
\end{table}

Exemplo tabela 2

\begin{table}
  \begin{tabular}{c r @{.} l}
    Expressão Pi &
    \multicolumn{2}{c}{Valor} \\
    \hline
    $\pi$ & 3 & 1416 \\
    $\pi^{\pi}$ & 36 & 46 \\
  \end{tabular}
\end{table}


\end{document}
